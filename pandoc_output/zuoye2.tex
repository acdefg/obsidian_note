% Options for packages loaded elsewhere
\PassOptionsToPackage{unicode}{hyperref}
\PassOptionsToPackage{hyphens}{url}
%
\documentclass[
]{article}
\usepackage{amsmath,amssymb}
\usepackage{lmodern}
\usepackage{iftex}
\ifPDFTeX
  \usepackage[T1]{fontenc}
  \usepackage[utf8]{inputenc}
  \usepackage{textcomp} % provide euro and other symbols
\else % if luatex or xetex
  \usepackage{unicode-math}
  \defaultfontfeatures{Scale=MatchLowercase}
  \defaultfontfeatures[\rmfamily]{Ligatures=TeX,Scale=1}
\fi
% Use upquote if available, for straight quotes in verbatim environments
\IfFileExists{upquote.sty}{\usepackage{upquote}}{}
\IfFileExists{microtype.sty}{% use microtype if available
  \usepackage[]{microtype}
  \UseMicrotypeSet[protrusion]{basicmath} % disable protrusion for tt fonts
}{}
\makeatletter
\@ifundefined{KOMAClassName}{% if non-KOMA class
  \IfFileExists{parskip.sty}{%
    \usepackage{parskip}
  }{% else
    \setlength{\parindent}{0pt}
    \setlength{\parskip}{6pt plus 2pt minus 1pt}}
}{% if KOMA class
  \KOMAoptions{parskip=half}}
\makeatother
\usepackage{xcolor}
\usepackage{color}
\usepackage{fancyvrb}
\newcommand{\VerbBar}{|}
\newcommand{\VERB}{\Verb[commandchars=\\\{\}]}
\DefineVerbatimEnvironment{Highlighting}{Verbatim}{commandchars=\\\{\}}
% Add ',fontsize=\small' for more characters per line
\newenvironment{Shaded}{}{}
\newcommand{\AlertTok}[1]{\textcolor[rgb]{1.00,0.00,0.00}{\textbf{#1}}}
\newcommand{\AnnotationTok}[1]{\textcolor[rgb]{0.38,0.63,0.69}{\textbf{\textit{#1}}}}
\newcommand{\AttributeTok}[1]{\textcolor[rgb]{0.49,0.56,0.16}{#1}}
\newcommand{\BaseNTok}[1]{\textcolor[rgb]{0.25,0.63,0.44}{#1}}
\newcommand{\BuiltInTok}[1]{\textcolor[rgb]{0.00,0.50,0.00}{#1}}
\newcommand{\CharTok}[1]{\textcolor[rgb]{0.25,0.44,0.63}{#1}}
\newcommand{\CommentTok}[1]{\textcolor[rgb]{0.38,0.63,0.69}{\textit{#1}}}
\newcommand{\CommentVarTok}[1]{\textcolor[rgb]{0.38,0.63,0.69}{\textbf{\textit{#1}}}}
\newcommand{\ConstantTok}[1]{\textcolor[rgb]{0.53,0.00,0.00}{#1}}
\newcommand{\ControlFlowTok}[1]{\textcolor[rgb]{0.00,0.44,0.13}{\textbf{#1}}}
\newcommand{\DataTypeTok}[1]{\textcolor[rgb]{0.56,0.13,0.00}{#1}}
\newcommand{\DecValTok}[1]{\textcolor[rgb]{0.25,0.63,0.44}{#1}}
\newcommand{\DocumentationTok}[1]{\textcolor[rgb]{0.73,0.13,0.13}{\textit{#1}}}
\newcommand{\ErrorTok}[1]{\textcolor[rgb]{1.00,0.00,0.00}{\textbf{#1}}}
\newcommand{\ExtensionTok}[1]{#1}
\newcommand{\FloatTok}[1]{\textcolor[rgb]{0.25,0.63,0.44}{#1}}
\newcommand{\FunctionTok}[1]{\textcolor[rgb]{0.02,0.16,0.49}{#1}}
\newcommand{\ImportTok}[1]{\textcolor[rgb]{0.00,0.50,0.00}{\textbf{#1}}}
\newcommand{\InformationTok}[1]{\textcolor[rgb]{0.38,0.63,0.69}{\textbf{\textit{#1}}}}
\newcommand{\KeywordTok}[1]{\textcolor[rgb]{0.00,0.44,0.13}{\textbf{#1}}}
\newcommand{\NormalTok}[1]{#1}
\newcommand{\OperatorTok}[1]{\textcolor[rgb]{0.40,0.40,0.40}{#1}}
\newcommand{\OtherTok}[1]{\textcolor[rgb]{0.00,0.44,0.13}{#1}}
\newcommand{\PreprocessorTok}[1]{\textcolor[rgb]{0.74,0.48,0.00}{#1}}
\newcommand{\RegionMarkerTok}[1]{#1}
\newcommand{\SpecialCharTok}[1]{\textcolor[rgb]{0.25,0.44,0.63}{#1}}
\newcommand{\SpecialStringTok}[1]{\textcolor[rgb]{0.73,0.40,0.53}{#1}}
\newcommand{\StringTok}[1]{\textcolor[rgb]{0.25,0.44,0.63}{#1}}
\newcommand{\VariableTok}[1]{\textcolor[rgb]{0.10,0.09,0.49}{#1}}
\newcommand{\VerbatimStringTok}[1]{\textcolor[rgb]{0.25,0.44,0.63}{#1}}
\newcommand{\WarningTok}[1]{\textcolor[rgb]{0.38,0.63,0.69}{\textbf{\textit{#1}}}}
\usepackage{graphicx}
\makeatletter
\def\maxwidth{\ifdim\Gin@nat@width>\linewidth\linewidth\else\Gin@nat@width\fi}
\def\maxheight{\ifdim\Gin@nat@height>\textheight\textheight\else\Gin@nat@height\fi}
\makeatother
% Scale images if necessary, so that they will not overflow the page
% margins by default, and it is still possible to overwrite the defaults
% using explicit options in \includegraphics[width, height, ...]{}
\setkeys{Gin}{width=\maxwidth,height=\maxheight,keepaspectratio}
% Set default figure placement to htbp
\makeatletter
\def\fps@figure{htbp}
\makeatother
\setlength{\emergencystretch}{3em} % prevent overfull lines
\providecommand{\tightlist}{%
  \setlength{\itemsep}{0pt}\setlength{\parskip}{0pt}}
\setcounter{secnumdepth}{-\maxdimen} % remove section numbering
\ifLuaTeX
  \usepackage{selnolig}  % disable illegal ligatures
\fi
\IfFileExists{bookmark.sty}{\usepackage{bookmark}}{\usepackage{hyperref}}
\IfFileExists{xurl.sty}{\usepackage{xurl}}{} % add URL line breaks if available
\urlstyle{same} % disable monospaced font for URLs
\hypersetup{
  hidelinks,
  pdfcreator={LaTeX via pandoc}}

\author{}
\date{}

\begin{document}

\hypertarget{ux5f00ux53d1ux73afux5883ux8bf4ux660e}{%
\subsection{开发环境说明}\label{ux5f00ux53d1ux73afux5883ux8bf4ux660e}}

\begin{quote}
操作系统:windows11 编译环境:SEGGER Embedded Studio for ARM
v6.34a(64-bit) ARM架构:v4T ARM Core Type:ARM9
\end{quote}

\hypertarget{ux601dux8defux5206ux6790}{%
\subsection{思路分析}\label{ux601dux8defux5206ux6790}}

因为ARM9嵌入式使用C语言编程,所以先在不使用各种预编译的图像处理的库函数的前提下,通过C语言编程实现图像旋转,再通过SEGGER仿真,测试程序性能。
ARM9可以直接读入RAW格式的图片,所以可以生成128x128(8bit)的RAW格式的灰度图像作为范例进行处理。

\hypertarget{ux56feux50cfux5904ux7406}{%
\subsubsection{图像处理}\label{ux56feux50cfux5904ux7406}}

\hypertarget{ux65b9ux6cd5ux4e00}{%
\paragraph{方法一}\label{ux65b9ux6cd5ux4e00}}

windows系统中的画图工具可以直接修改图片的大小以及格式,将其保存为8bit的BMP格式,之后再通过C语言程序将其转化为RAW格式。

\hypertarget{ux65b9ux6cd5ux4e8c}{%
\paragraph{方法二}\label{ux65b9ux6cd5ux4e8c}}

可以直接通过PHOTOSHOP软件更改图片颜色信息、图片格式和图片像素大小,直接生成128x128的RAW格式灰度图像(8bit)。

\hypertarget{ux56feux50cfux65cbux8f6cux7b97ux6cd5}{%
\subsubsection{图像旋转算法}\label{ux56feux50cfux65cbux8f6cux7b97ux6cd5}}

通过查阅资料,发现可以通过直接插值,最邻近插值,双线性内插值,以及双立方(三次)卷积法等实现图像旋转,而其中双线性插值计算有着较低的复杂度和较高的照片质量,对于一款性能有限的ARM9芯片来说,使用双线性插值计算的性价比较高.

\hypertarget{ux57faux672cux539fux7406}{%
\paragraph{基本原理}\label{ux57faux672cux539fux7406}}

\hypertarget{ux56feux7247ux65cbux8f6cux539fux7406}{%
\subparagraph{图片旋转原理}\label{ux56feux7247ux65cbux8f6cux539fux7406}}

首先,RAW图片的信息都以像素值的方式存储在每个像素点上,而128x128的8bit灰度图像,在计算机中存储方式是一个128x128的数组,其中每个数据的大小为一字节。
\includegraphics{https://raw.githubusercontent.com/acdefg/cdn/main/obsidian/20220926222510.png}
将一幅图片旋转45度相当于将数组中的数据旋转45度,下图使用5x5的数组举例
\includegraphics{https://raw.githubusercontent.com/acdefg/cdn/main/obsidian/20220926222609.png}
若将其旋转45度之后会形成下图所示的数组,其中蓝色的0表示新形成的空位。易得矩阵围绕中心点旋转,将矩阵旋转会形成新的空白区域扩大数组大小,扩大的大小可以用几何方法求解。
\includegraphics{https://raw.githubusercontent.com/acdefg/cdn/main/obsidian/20220926222735.png}
设height为图片高度,width为图片宽度,则新生成的图片边长为 \textbar{}
height x cos(θ) \textbar{} + \textbar{} width x sin(θ) \textbar{}
向上取整。 此时,设原图像坐标为(Xp, Yp),中心点坐标为(cenX\_p,
cenY\_p)新图像高度为heightf,宽度为widthf,像素坐标为(Xf,
Yf)则原图像像素坐标和新图像像素坐标的对应关系为: \textgreater xp =
cos(θ) x Xf - sin(θ) x Yf + cenX\_p; yp = cos(θ) x Xf + sin(θ) x Yf +
cenX\_p;

这样就会导致会计算出来的点并非实数,若直接用这种方式取整计算,会使得图片质量下降。

\hypertarget{ux53ccux7ebfux6027ux63d2ux503cux539fux7406}{%
\subparagraph{双线性插值原理}\label{ux53ccux7ebfux6027ux63d2ux503cux539fux7406}}

若计算新图像中的(1,0)点对应原图像的坐标会得到(65.14, 65.14)
双线性插值则是取计算结果附近的四个点,通过加权的方式将四个点的像素组合起来,得到当前位置点的像素值。
如上例可以表示为(65+0.14, 65+0.14),即(i+u, j+v),
i,j为坐标值整数部分;u,v为坐标值小数部分,
计算公式可以表示为:f(i+u,j+v) = (1-u)(1-v)f(i,j) + (1-u)vf(i,j+1) +
u(1-v)f(i+1,j) + uvf(i+1,j+1)
以点之间的横纵坐标的乘积距离作为权重计算当前点的像素值。

\hypertarget{ux6027ux80fdux4f18ux5316}{%
\paragraph{性能优化}\label{ux6027ux80fdux4f18ux5316}}

上述公式在计算过程中会涉及到较多的浮点数乘法运算,浮点数乘法运算相较于整数乘法会更为复杂,若将浮点数运算转化为整数运算,计算速度会有所提升。

\begin{Shaded}
\begin{Highlighting}[]
\CommentTok{/*}
\CommentTok{将浮点数扩大amp倍,化成整数,减少浮点数的运算}
\CommentTok{ */}
\DataTypeTok{int}\NormalTok{ u }\OperatorTok{=} \OperatorTok{(}\NormalTok{fXp }\OperatorTok{{-}}\NormalTok{ Xp}\OperatorTok{)*}\NormalTok{amp}\OperatorTok{;} \CommentTok{//浮点坐标小数部分,左移cout位}
\DataTypeTok{int}\NormalTok{ v }\OperatorTok{=} \OperatorTok{(}\NormalTok{fYp }\OperatorTok{{-}}\NormalTok{ Yp}\OperatorTok{)*}\NormalTok{amp}\OperatorTok{;} 
\CommentTok{//printf("\%d \%d\textbackslash{}n", u, v);}
\CommentTok{/*}
\CommentTok{f(i+u,j+v) = (1{-}u)(1{-}v)f(i,j) + (1{-}u)vf(i,j+1) + u(1{-}v)f(i+1,j) + uvf(i+1,j+1)}
\CommentTok{再右移7+7=14位,还原原来扩大的倍数}
\CommentTok{ */}
\ControlFlowTok{if}\OperatorTok{(}\NormalTok{Xp }\OperatorTok{\textgreater{}=} \DecValTok{0} \OperatorTok{\&\&}\NormalTok{ Xp }\OperatorTok{\textless{}}\NormalTok{ height }\OperatorTok{\&\&}\NormalTok{ Yp }\OperatorTok{\textgreater{}=} \DecValTok{0} \OperatorTok{\&\&}\NormalTok{ Yp }\OperatorTok{\textless{}}\NormalTok{ width}\OperatorTok{)} \OperatorTok{\{}\CommentTok{//在原图范围内}
\NormalTok{  outimg}\OperatorTok{[}\NormalTok{i }\OperatorTok{*}\NormalTok{ widthf }\OperatorTok{+}\NormalTok{ j}\OperatorTok{]} \OperatorTok{=} \OperatorTok{(} \OperatorTok{(}\NormalTok{amp}\OperatorTok{{-}}\NormalTok{u}\OperatorTok{)*(}\NormalTok{amp}\OperatorTok{{-}}\NormalTok{v}\OperatorTok{)*}\NormalTok{   orimg}\OperatorTok{[}\NormalTok{Xp }\OperatorTok{*}\NormalTok{ width }\OperatorTok{+}\NormalTok{ Yp}\OperatorTok{]} \OperatorTok{+} 
                             \OperatorTok{(}\NormalTok{amp}\OperatorTok{{-}}\NormalTok{u}\OperatorTok{)*}\NormalTok{v}\OperatorTok{*}\NormalTok{  orimg}\OperatorTok{[}\NormalTok{Xp }\OperatorTok{*}\NormalTok{ width }\OperatorTok{+}\NormalTok{ Yp}\OperatorTok{+(}\NormalTok{Yp }\OperatorTok{==}\NormalTok{ width }\OperatorTok{{-}} \DecValTok{1} \OperatorTok{?} \DecValTok{0} \OperatorTok{:} \DecValTok{1} \OperatorTok{)]} \OperatorTok{+} 
\NormalTok{                             u}\OperatorTok{*(}\NormalTok{amp}\OperatorTok{{-}}\NormalTok{v}\OperatorTok{)*}\NormalTok{  orimg}\OperatorTok{[(}\NormalTok{Xp}\OperatorTok{+(}\NormalTok{Xp }\OperatorTok{==}\NormalTok{ height }\OperatorTok{{-}} \DecValTok{1} \OperatorTok{?} \DecValTok{0} \OperatorTok{:} \DecValTok{1} \OperatorTok{))} \OperatorTok{*}\NormalTok{ width }\OperatorTok{+} 
\NormalTok{                             Yp}\OperatorTok{]} \OperatorTok{+} 
\NormalTok{                             u}\OperatorTok{*}\NormalTok{v}\OperatorTok{*}\NormalTok{         orimg}\OperatorTok{[(}\NormalTok{Xp}\OperatorTok{+(}\NormalTok{Xp }\OperatorTok{==}\NormalTok{ height }\OperatorTok{{-}} \DecValTok{1} \OperatorTok{?} \DecValTok{0} \OperatorTok{:} \DecValTok{1} \OperatorTok{))} \OperatorTok{*}\NormalTok{ width }\OperatorTok{+} 
\NormalTok{                             Yp}\OperatorTok{+(}\NormalTok{Yp }\OperatorTok{==}\NormalTok{ width }\OperatorTok{{-}} \DecValTok{1} \OperatorTok{?} \DecValTok{0} \OperatorTok{:} \DecValTok{1} \OperatorTok{)]} \OperatorTok{)} \OperatorTok{\textgreater{}\textgreater{}} \OperatorTok{(}\NormalTok{cout}\OperatorTok{*}\DecValTok{2}\OperatorTok{);}
\end{Highlighting}
\end{Shaded}

先将浮点值u,v扩大amp倍,amp =
2\^{}cout,在计算完成后再将计算结果右移cout x
2位,可以一定程度上加快程序性能。

\hypertarget{ux7a0bux5e8fux4effux771f}{%
\subsubsection{程序仿真}\label{ux7a0bux5e8fux4effux771f}}

在SEGGER上面仿真:
在SEGGER工程中新建文件夹存放需要读入的图片,与KEIL不同的是,SEGGER可以自动通过程序读入图片并实现输出。
\includegraphics{https://raw.githubusercontent.com/acdefg/cdn/main/obsidian/20220927145047.png}

\hypertarget{ux4effux771fux7ed3ux679c}{%
\paragraph{仿真结果}\label{ux4effux771fux7ed3ux679c}}

使用的图片和旋转后的图片对比:
\includegraphics{https://raw.githubusercontent.com/acdefg/cdn/main/obsidian/20220927144736.png}

\includegraphics{https://raw.githubusercontent.com/acdefg/cdn/main/obsidian/20220927144816.png}
空余部分程序缺省为0,像素表现为黑色。

\hypertarget{ux4effux771fux7ed3ux679cux5206ux6790}{%
\paragraph{仿真结果分析}\label{ux4effux771fux7ed3ux679cux5206ux6790}}

对于一款给定的芯片,在编写类似的程序时,若想提升程序的运行性能,有时可以从程序编写、空间占用以及算法运算等软件层面入手,改善程序的运行速度。

\hypertarget{ux7a0bux5e8fux6e90ux7801}{%
\subsubsection{程序源码}\label{ux7a0bux5e8fux6e90ux7801}}

\begin{Shaded}
\begin{Highlighting}[]
\PreprocessorTok{\#include }\ImportTok{\textless{}stdio.h\textgreater{}}
\PreprocessorTok{\#include }\ImportTok{\textless{}stdlib.h\textgreater{}}
\PreprocessorTok{\#include }\ImportTok{\textless{}math.h\textgreater{}}
\PreprocessorTok{\#include}\ImportTok{\textless{}time.h\textgreater{}}\PreprocessorTok{ }

\PreprocessorTok{\#define height  128    }
\PreprocessorTok{\#define width   128 }
\PreprocessorTok{\#define Angle   45                  }\CommentTok{// (adjustable) rotate angle}
\PreprocessorTok{\#define heightf 182                 }\CommentTok{// 向上取整fabs(height*cos(angle)) }
\PreprocessorTok{\#define widthf  182                 }

\KeywordTok{typedef} \DataTypeTok{unsigned} \DataTypeTok{char}\NormalTok{  BYTE}\OperatorTok{;}        \CommentTok{// define BYTE type,1 char, 8 bit, to save space}

\DataTypeTok{char} \OperatorTok{*}\NormalTok{p }\OperatorTok{=} \StringTok{".}\SpecialCharTok{\textbackslash{}\textbackslash{}}\StringTok{asset}\SpecialCharTok{\textbackslash{}\textbackslash{}}\StringTok{2.raw"}\OperatorTok{;}        \CommentTok{// input image path}
\DataTypeTok{char} \OperatorTok{*}\NormalTok{q }\OperatorTok{=} \StringTok{".}\SpecialCharTok{\textbackslash{}\textbackslash{}}\StringTok{asset}\SpecialCharTok{\textbackslash{}\textbackslash{}}\StringTok{2\_out1.raw"}\OperatorTok{;}    \CommentTok{// output image path}

\DataTypeTok{int}\NormalTok{ main}\OperatorTok{()\{}
    
\NormalTok{    clock\_t start}\OperatorTok{,}\NormalTok{finish}\OperatorTok{;} \CommentTok{// 定义变量}
    \DataTypeTok{double}\NormalTok{ time}\OperatorTok{;}            \CommentTok{// }
\NormalTok{    start }\OperatorTok{=}\NormalTok{ clock}\OperatorTok{();} 
    \CommentTok{//printf("/************************START************************/\textbackslash{}n");}
    \CommentTok{//printf("Aim at rotating the certain image 45 degrees\textbackslash{}n");}
    
    \CommentTok{/*}
\CommentTok{    Usage: input image}
\CommentTok{    Varibles:   orimg:  store the original image data}
\CommentTok{                p:      input image path(pointer) }

\CommentTok{     */}
    \DataTypeTok{FILE} \OperatorTok{*}\NormalTok{fp }\OperatorTok{=}\NormalTok{ NULL}\OperatorTok{;}  
  
\NormalTok{    BYTE orimg}\OperatorTok{[}\NormalTok{height }\OperatorTok{*}\NormalTok{ width}\OperatorTok{];}
    \CommentTok{// 打开raw图像文件  }
    \ControlFlowTok{if}\OperatorTok{((}\NormalTok{fp }\OperatorTok{=}\NormalTok{ fopen}\OperatorTok{(}\NormalTok{ p}\OperatorTok{,} \StringTok{"rb"} \OperatorTok{))} \OperatorTok{==}\NormalTok{ NULL}\OperatorTok{)} \OperatorTok{\{}  
\NormalTok{        printf}\OperatorTok{(}\StringTok{"can not open the image}\SpecialCharTok{\textbackslash{}n}\StringTok{ "} \OperatorTok{);}  
        \ControlFlowTok{return} \DecValTok{0}\OperatorTok{;}  
    \OperatorTok{\}} \ControlFlowTok{else} \OperatorTok{\{}  
\NormalTok{        printf}\OperatorTok{(}\StringTok{"Already opened the image from \%s.}\SpecialCharTok{\textbackslash{}n}\StringTok{"}\OperatorTok{,}\NormalTok{ p}\OperatorTok{);}  
    \OperatorTok{\}} 

    \DataTypeTok{int}\NormalTok{ i}\OperatorTok{,}\NormalTok{ j}\OperatorTok{;}
    \ControlFlowTok{for}\OperatorTok{(}\NormalTok{ i }\OperatorTok{=} \DecValTok{0}\OperatorTok{;}\NormalTok{ i }\OperatorTok{\textless{}}\NormalTok{ height}\OperatorTok{;}\NormalTok{ i}\OperatorTok{++} \OperatorTok{)} \OperatorTok{\{}  
        \CommentTok{// 逐行读入图像数据}
\NormalTok{        fread}\OperatorTok{(} \OperatorTok{\&}\NormalTok{orimg}\OperatorTok{[}\NormalTok{i}\OperatorTok{*}\NormalTok{width}\OperatorTok{],} \DecValTok{1} \OperatorTok{,}\NormalTok{ width}\OperatorTok{,}\NormalTok{ fp }\OperatorTok{);} 
    \OperatorTok{\}}
\NormalTok{    fclose}\OperatorTok{(}\NormalTok{fp}\OperatorTok{);}

    \CommentTok{/*}
\CommentTok{    Usage:  rotate image}
\CommentTok{    Note:   采用双线性插值法}

\CommentTok{     */}
    \CommentTok{//printf("Start rotating.\textbackslash{}n");}

    \DataTypeTok{int}\NormalTok{ cenX\_p}\OperatorTok{,}\NormalTok{ cenY\_p}\OperatorTok{,}\NormalTok{ cenX\_f}\OperatorTok{,}\NormalTok{ cenY\_f}\OperatorTok{;}     \CommentTok{//旋转前后的中心点的坐标}
\NormalTok{    cenX\_p }\OperatorTok{=}\NormalTok{ width }\OperatorTok{/} \DecValTok{2}\OperatorTok{;}                     \CommentTok{// 64}
\NormalTok{    cenY\_p }\OperatorTok{=}\NormalTok{ height }\OperatorTok{/} \DecValTok{2}\OperatorTok{;}                    \CommentTok{// 64}

    \DataTypeTok{int}\NormalTok{     Xp}\OperatorTok{,}\NormalTok{ Yp}\OperatorTok{,}\NormalTok{ Xf}\OperatorTok{,}\NormalTok{ Yf}\OperatorTok{;}                 \CommentTok{//旋转前后对应的像素点整数坐标}
    \DataTypeTok{double}\NormalTok{  fXp}\OperatorTok{,}\NormalTok{ fYp}\OperatorTok{;}                       \CommentTok{//对应的浮点坐标(计算所得浮点数坐标)}
    \DataTypeTok{double}\NormalTok{  angle }\OperatorTok{=} \OperatorTok{(}\DataTypeTok{double}\OperatorTok{)}\FloatTok{1.0} \OperatorTok{*}\NormalTok{ Angle }\OperatorTok{*} \FloatTok{3.1415926} \OperatorTok{/} \DecValTok{180}\OperatorTok{;}
\NormalTok{    cenX\_f }\OperatorTok{=}\NormalTok{ widthf }\OperatorTok{/} \DecValTok{2}\OperatorTok{;}
\NormalTok{    cenY\_f }\OperatorTok{=}\NormalTok{ heightf }\OperatorTok{/} \DecValTok{2}\OperatorTok{;}

    \DataTypeTok{int}\NormalTok{ amp}\OperatorTok{,}\NormalTok{ cout}\OperatorTok{;}
\NormalTok{    cout   }\OperatorTok{=} \DecValTok{11}\OperatorTok{;}
\NormalTok{    amp    }\OperatorTok{=} \DecValTok{2048}\OperatorTok{;}
    \CommentTok{//amp    = pow(2, cout);}
    \CommentTok{//printf("cout: \%d , amp: \%d\textbackslash{}n", cout, amp);}

\NormalTok{    BYTE outimg}\OperatorTok{[}\NormalTok{heightf }\OperatorTok{*}\NormalTok{ widthf}\OperatorTok{];}
    \ControlFlowTok{for}\OperatorTok{(}\NormalTok{i }\OperatorTok{=} \DecValTok{0}\OperatorTok{;}\NormalTok{i }\OperatorTok{\textless{}}\NormalTok{ heightf}\OperatorTok{;}\NormalTok{ i}\OperatorTok{++)} \OperatorTok{\{}
        \ControlFlowTok{for}\OperatorTok{(}\NormalTok{j }\OperatorTok{=} \DecValTok{0}\OperatorTok{;}\NormalTok{j }\OperatorTok{\textless{}}\NormalTok{ widthf}\OperatorTok{;}\NormalTok{ j}\OperatorTok{++)} \OperatorTok{\{}
\NormalTok{            outimg}\OperatorTok{[}\NormalTok{i }\OperatorTok{*}\NormalTok{ widthf }\OperatorTok{+}\NormalTok{ j}\OperatorTok{]} \OperatorTok{=} \DecValTok{0}\OperatorTok{;}
\NormalTok{            Xf  }\OperatorTok{=}\NormalTok{    i }\OperatorTok{{-}}\NormalTok{ cenX\_f}\OperatorTok{;}
\NormalTok{            Yf  }\OperatorTok{=}\NormalTok{    j }\OperatorTok{{-}}\NormalTok{ cenY\_f}\OperatorTok{;}
\NormalTok{            fXp }\OperatorTok{=}\NormalTok{    cos}\OperatorTok{(}\NormalTok{angle}\OperatorTok{)} \OperatorTok{*}\NormalTok{ Xf }\OperatorTok{{-}}\NormalTok{ sin}\OperatorTok{(}\NormalTok{angle}\OperatorTok{)} \OperatorTok{*}\NormalTok{ Yf }\OperatorTok{+}\NormalTok{ cenX\_p}\OperatorTok{;}   \CommentTok{//对应原图横坐标}
\NormalTok{            fYp }\OperatorTok{=}\NormalTok{    sin}\OperatorTok{(}\NormalTok{angle}\OperatorTok{)} \OperatorTok{*}\NormalTok{ Xf }\OperatorTok{+}\NormalTok{ cos}\OperatorTok{(}\NormalTok{angle}\OperatorTok{)} \OperatorTok{*}\NormalTok{ Yf }\OperatorTok{+}\NormalTok{ cenY\_p}\OperatorTok{;}   \CommentTok{//对应原图纵坐标}
\NormalTok{            Xp  }\OperatorTok{=}    \OperatorTok{(}\DataTypeTok{int}\OperatorTok{)}\NormalTok{fXp}\OperatorTok{;}
\NormalTok{            Yp  }\OperatorTok{=}    \OperatorTok{(}\DataTypeTok{int}\OperatorTok{)}\NormalTok{fYp}\OperatorTok{;}
            \CommentTok{//printf("\%f  \%f\textbackslash{}n", fXp {-} Xp, fYp {-} Yp);}
            \CommentTok{/*}
\CommentTok{            将浮点数扩大amp倍,化成整数,减少浮点数的运算}
\CommentTok{             */}
            \DataTypeTok{int}\NormalTok{ u }\OperatorTok{=} \OperatorTok{(}\NormalTok{fXp }\OperatorTok{{-}}\NormalTok{ Xp}\OperatorTok{)*}\NormalTok{amp}\OperatorTok{;} \CommentTok{//浮点坐标小数部分,左移7位}
            \DataTypeTok{int}\NormalTok{ v }\OperatorTok{=} \OperatorTok{(}\NormalTok{fYp }\OperatorTok{{-}}\NormalTok{ Yp}\OperatorTok{)*}\NormalTok{amp}\OperatorTok{;} 
            \CommentTok{//printf("\%d \%d\textbackslash{}n", u, v);}
            \CommentTok{/*}
\CommentTok{            f(i+u,j+v) = (1{-}u)(1{-}v)f(i,j) + (1{-}u)vf(i,j+1) + u(1{-}v)f(i+1,j) + }
\CommentTok{            uvf(i+1,j+1)}
\CommentTok{            再右移7+7=14位,还原原来扩大的倍数}
\CommentTok{             */}
            \ControlFlowTok{if}\OperatorTok{(}\NormalTok{Xp }\OperatorTok{\textgreater{}=} \DecValTok{0} \OperatorTok{\&\&}\NormalTok{ Xp }\OperatorTok{\textless{}}\NormalTok{ height }\OperatorTok{\&\&}\NormalTok{ Yp }\OperatorTok{\textgreater{}=} \DecValTok{0} \OperatorTok{\&\&}\NormalTok{ Yp }\OperatorTok{\textless{}}\NormalTok{ width}\OperatorTok{)} \OperatorTok{\{}\CommentTok{//在原图范围内}
\NormalTok{              outimg}\OperatorTok{[}\NormalTok{i }\OperatorTok{*}\NormalTok{ widthf }\OperatorTok{+}\NormalTok{ j}\OperatorTok{]} \OperatorTok{=} \OperatorTok{(} \OperatorTok{(}\NormalTok{amp}\OperatorTok{{-}}\NormalTok{u}\OperatorTok{)*(}\NormalTok{amp}\OperatorTok{{-}}\NormalTok{v}\OperatorTok{)*}\NormalTok{   orimg}\OperatorTok{[}\NormalTok{Xp }\OperatorTok{*}\NormalTok{ width }\OperatorTok{+}\NormalTok{ Yp}\OperatorTok{]} \OperatorTok{+} 
              \OperatorTok{(}\NormalTok{amp}\OperatorTok{{-}}\NormalTok{u}\OperatorTok{)*}\NormalTok{v}\OperatorTok{*}\NormalTok{ orimg}\OperatorTok{[}\NormalTok{Xp }\OperatorTok{*}\NormalTok{ width }\OperatorTok{+}\NormalTok{ Yp}\OperatorTok{+(}\NormalTok{Yp }\OperatorTok{==}\NormalTok{ width }\OperatorTok{{-}} \DecValTok{1} \OperatorTok{?} \DecValTok{0} \OperatorTok{:} \DecValTok{1} \OperatorTok{)]} \OperatorTok{+} 
\NormalTok{              u}\OperatorTok{*(}\NormalTok{amp}\OperatorTok{{-}}\NormalTok{v}\OperatorTok{)*}\NormalTok{ orimg}\OperatorTok{[(}\NormalTok{Xp}\OperatorTok{+(}\NormalTok{Xp }\OperatorTok{==}\NormalTok{ height }\OperatorTok{{-}} \DecValTok{1} \OperatorTok{?} \DecValTok{0} \OperatorTok{:} \DecValTok{1} \OperatorTok{))} \OperatorTok{*}\NormalTok{ width }\OperatorTok{+}\NormalTok{ Yp}\OperatorTok{]} \OperatorTok{+} 
\NormalTok{              u}\OperatorTok{*}\NormalTok{v}\OperatorTok{*}\NormalTok{       orimg}\OperatorTok{[(}\NormalTok{Xp}\OperatorTok{+(}\NormalTok{Xp }\OperatorTok{==}\NormalTok{ height }\OperatorTok{{-}} \DecValTok{1} \OperatorTok{?} \DecValTok{0} \OperatorTok{:} \DecValTok{1} \OperatorTok{))} \OperatorTok{*}\NormalTok{ width }\OperatorTok{+}\NormalTok{ Yp}\OperatorTok{+(}\NormalTok{Yp }\OperatorTok{==} 
\NormalTok{              width }\OperatorTok{{-}} \DecValTok{1} \OperatorTok{?} \DecValTok{0} \OperatorTok{:} \DecValTok{1} \OperatorTok{)]} \OperatorTok{)} \OperatorTok{\textgreater{}\textgreater{}} \OperatorTok{(}\NormalTok{cout}\OperatorTok{*}\DecValTok{2}\OperatorTok{);}
            \OperatorTok{\}}
        \OperatorTok{\}}
        \CommentTok{//rotate部分计算很慢,可以打印进度作为提示}
        \CommentTok{//if(i\%18==0)\{printf("Rotating...\%d\%\textbackslash{}n", i/18*10);\}}
    \OperatorTok{\}}
    \CommentTok{//printf("Finish rotating.\textbackslash{}n");}

    \CommentTok{/*}
\CommentTok{    Usage: output image}
\CommentTok{    Varibles:   outimg:  store the image data used for output}
\CommentTok{                q:       output image path(pointer) }

\CommentTok{     */}
    \ControlFlowTok{if}\OperatorTok{(} \OperatorTok{(}\NormalTok{ fp }\OperatorTok{=}\NormalTok{ fopen}\OperatorTok{(}\NormalTok{ q}\OperatorTok{,} \StringTok{"wb"} \OperatorTok{)} \OperatorTok{)} \OperatorTok{==}\NormalTok{ NULL }\OperatorTok{)}  
    \OperatorTok{\{}  
\NormalTok{        printf}\OperatorTok{(}\StringTok{"Can\textquotesingle{}t create the image at \%s}\SpecialCharTok{\textbackslash{}n}\StringTok{"}\OperatorTok{,}\NormalTok{ q}\OperatorTok{);}  
        \ControlFlowTok{return} \DecValTok{0}\OperatorTok{;}  
    \OperatorTok{\}}  
      
    \ControlFlowTok{for}\OperatorTok{(}\NormalTok{ i }\OperatorTok{=} \DecValTok{0}\OperatorTok{;}\NormalTok{ i }\OperatorTok{\textless{}}\NormalTok{ heightf}\OperatorTok{;}\NormalTok{ i}\OperatorTok{++} \OperatorTok{)} \OperatorTok{\{}  
\NormalTok{        fwrite}\OperatorTok{(} \OperatorTok{\&}\NormalTok{outimg}\OperatorTok{[}\NormalTok{i}\OperatorTok{*}\NormalTok{widthf}\OperatorTok{],} \DecValTok{1} \OperatorTok{,}\NormalTok{ widthf}\OperatorTok{,}\NormalTok{ fp }\OperatorTok{);}  \CommentTok{//按行输出}
    \OperatorTok{\}} 
\NormalTok{    printf}\OperatorTok{(}\StringTok{"The image is written to \%s}\SpecialCharTok{\textbackslash{}n}\StringTok{"}\OperatorTok{,}\NormalTok{ q}\OperatorTok{);}  
\NormalTok{    fclose}\OperatorTok{(}\NormalTok{fp}\OperatorTok{);}

\NormalTok{    finish}\OperatorTok{=}\NormalTok{clock}\OperatorTok{();}    \CommentTok{//结束}
\NormalTok{    time}\OperatorTok{=(}\DataTypeTok{double}\OperatorTok{)(}\NormalTok{finish}\OperatorTok{{-}}\NormalTok{start}\OperatorTok{)/}\NormalTok{CLOCKS\_PER\_SEC}\OperatorTok{;}\CommentTok{//计算运行时间}
\NormalTok{    printf}\OperatorTok{(}\StringTok{"time=\%lf}\SpecialCharTok{\textbackslash{}n}\StringTok{"}\OperatorTok{,}\NormalTok{time}\OperatorTok{);}\CommentTok{//输出运行时间}

\OperatorTok{\}}
\end{Highlighting}
\end{Shaded}

\hypertarget{ux53c2ux8003ux94feux63a5}{%
\subsubsection{参考链接}\label{ux53c2ux8003ux94feux63a5}}

\href{https://www.lvxuefei.top/\%E4\%BD\%BF\%E7\%94\%A8ARM-\%E5\%A4\%84\%E7\%90\%86\%E5\%99\%A8\%E9\%A1\%BA\%E6\%97\%B6\%E9\%92\%88\%E6\%97\%8B\%E8\%BD\%AC45\%E5\%BA\%A6\%E7\%81\%B0\%E5\%BA\%A6\%E5\%9B\%BE\%E5\%83\%8F/\#comments}{使用
ARM 处理器顺时针旋转 45 度灰度图像 \textbar{} 槐雪}
\href{https://blog.csdn.net/qq_37577735/article/details/80041586}{图像处理之双线性插值法\_Brandon懂你的博客-CSDN博客\_图像线性插值}
\href{https://blog.csdn.net/u012745229/article/details/51405332}{C语言\_左移(\textless\textless)和右移(\textgreater\textgreater)\_蓝海洋高飞的博客-CSDN博客\_c语言的左移函数}

\end{document}
